Können wir nur mit Lineal und Zirkel ein Quadrat aus einem gegebenem Kreis konstruieren? Die kurze Antwort, wie vielleicht schon vermutet wird, ist nein. Die Antwort greift jedoch viel zu kurz, denn versuchen wir zu erklären, warum diese geometrische Umformung unmöglich ist, stoßen wir auf einige Komplikationen. Wie beweist man solch ein Problem? Bevor wir diese Fragen klären, müssen wir zunächst noch einige Vorüberlegungen anstellen. \par
Die Popularität des Problems der Quadratur des Kreises ist wohl größer als die vieler anderer mathematischer Probleme. Nicht weil dessen Lösung, wenn sie denn möglich wäre, eine große Bedeutung für wissenschaftliche Errungenschaften hätte, sondern vielmehr aufgrund der Einfachheit des Problems. So bedient man sich bei der Quadratur des Kreises der geläufigen Figuren und Begriffe wie Flächen, Lineal und Zirkel. Das Problem hat sogar den Weg in den Sprachgebrauch mancher Menschen geschafft. Redewendungen wie „Die Quadratur des Kreises versuchen“ werden -- wie es im Duden notiert steht\footnote{Unter: https://www.duden.de/rechtschreibung/Quadratur, 1.a) Wendungen} -- benutzt, um die Unlösbarkeit eines Problems darzustellen. Zum Beispiel stand in der Ostthüringer Zeitung am 17.06.2014\footnote{Aus einer Präsentation der Universität Duisburg-Essen von Prof. Dr. Jan Kohlhaase, Folie 7 [\UniPresentationInt]}: ”Die Aufgabe, die Landrat Andreas Heller (CDU) in den kommenden Wochen zu lösen hat, kommt der Quadratur des Kreises gleich: Er muss nach der Kommunalwahl versuchen, eine stabile Mehrheit im Kreistag zu etablieren.“ Die Aussage, die damit verknüpft war, ist die, dass die Aufgabe des Landrates Andreas Heller, eine stabile Mehrheit im Kreistag zu erschaffen, gänzlich unmöglich ist. Warum diese Redewendung nahezu immer falsch benutzt wird, wird im Laufe der Arbeit noch geklärt. \par In vorliegender Seminararbeit soll hauptsächlich aufgezeigt werden, warum die Quadratur des Kreises unmöglich ist. Dazu werfen wir einen Blick auf die Mathematik, Begrifflichkeiten und Geometrie, die mit unserem Problem in Verbindung stehen. Widmen wir uns aber zunächst einer genauen Formulierung des Problems. \par
