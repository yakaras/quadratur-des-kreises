\usepackage[ngerman]{babel}
\usepackage[a4paper, hmargin={3cm, 2.5cm}, vmargin={3cm, 2.5cm}]{geometry}
\usepackage{tikz,xcolor,pict2e,subcaption,titelseite,amsthm,  wasysym }
\usepackage[sfdefault]{inter}
\usepackage{tikz}
\usepackage{amssymb}
\newcommand{\N}{\mathbb N}
\newcommand{\Q}{\mathbb Q}
\newcommand{\Z}{\mathbb Z}
\newcommand{\R}{\mathbb R}
\newcommand{\C}{\mathbb C}
\newcommand*\circled[1]{\tikz[baseline=(char.base)]{
            \node[shape=circle,draw,inner sep=2pt] (char) {#1};}}
            \usepackage[tbtags]{amsmath}
\usepackage{mathtools}  
\usepackage{fancyhdr}
\pagestyle{fancy}
\usepackage{graphicx}
\graphicspath{./images/} 
\usepackage{sidecap}
\usepackage[labelfont=footnotesize, textfont=footnotesize]{caption}
\usepackage{refcount}
\usepackage{comment}
\usepackage{wrapfig}
\usepackages{setspace}
\onehalfspace
\newtheorem{theorem}{Theorem}[section]
\newtheorem{corollary}{Corollary}[theorem]
\newtheorem{lemma}[theorem]{Lemma}
\newenvironment{simplechar}{%
   \catcode`\$=12
   \catcode`\&=12
   \catcode`\#=12
   \catcode`\^=12
   \catcode`\_=12
   \catcode`\~=12
   \catcode`\%=12
}{}

\renewcommand{\sectionmark}[1]{\markright{The Book Title}}
\newcommand{\EugenBeutelInt}{1}
\newcommand{\GoogleCloudInt}{14}
\newcommand{\UniPresentationInt}{13}
\newcommand{\WieleitnerInt}{11}
\newcommand{\MillaInt}{8}
\newcommand{\MohrInt}{9}
\newcommand{\BabineczInt}{2}
\newcommand{\ActaInt}{4}
\newcommand{\FritschInt}{3}
\newcommand{\PiUlm}{12}
\newcommand{\AnnalenInt}{7}
\newcommand{\MoserInt}{10}
\newcommand{\AlgebraInt}{5}
\newcommand{\HobsonInt}{6}

\newcommand{\ratio}{\glqq \textsl{ratio}\grqq{}}

\newtheorem{definition}{Definition}[section]


\fancyhf{}
\fancyhead[LE]{\makebox[1.5em][l]{\thepage}}
\fancyhead[RO]{\makebox[1.5em][r]{\thepage}}                                                                                                                    


 \renewcommand{\familydefault}{\sfdefault}
 
\titel{Quadratur des Kreises}
\autor{Johann Uhl} 
\date{2022}
\headline{Hans-Carossa-Gymnasium}
