Die Quadratur des Kreises ist zunächst nichts anderes als die exakte, zeichnerische Umformung einer Kreisfläche in ein flächeninhaltsgleiches Quadrat nur mit Hilfe von Lineal und Zirkel, die in einer endlichen Anzahl von Schritten erfolgen soll. Dabei kann man sich nur der Konstruktionsmöglichkeiten mit Lineal und Zirkel bedienen, so wie eine Gerade zu zeichnen oder einen Kreis mit gegebenem Radius zu konstruieren. Nimmt man als Beispiel einen Kreis mit dem Radius \(r=1\), dessen Flächen dann \(A=1^2\cdot\pi=\pi\)  wäre. Dann könnte man die Kreisquadratur also auch als eine exakte Konstruktion von \(\pi\) in Form von einer Quadratfläche mit der Kantenlänge \(\sqrt{\pi}\) beschreiben. Allgemein, wenn man von einer Kreisquadratur redet, ist dort immer ein Kreis mit Radius \(r = 1\) gemeint. Doch was verstehen wir eigentlich unter der Zahl \(\pi\)?\par
Die Kreiszahl beschreibt das Verhältnis des Kreisumfangs \(u\) zum Durchmesser \(d\), so ist \(u=\pi d\). Da nun aber alle Kreise ähnliche Figuren sind, ist für jeden Kreis die Konstante, um den Umfang mit dem Durchmesser oder Radius zu beschreiben immer gleich. Diese Konstante ist \(\pi\). Daraus ergeben sich die Formeln \(u=\pi d\) oder \(u=\pi 2r\), da    \(d=2r\). Der Wert der Zahl $\pi$ beträgt \[3,14159265358979323846\dots\] Da die Fläche eines Kreises mit der Formel $A=r^2\pi$ beschrieben werden kann, gibt $\pi$ auch das Verhältnis der Fläche eines Kreises und dessen Radius an.\footnote{E. Beutel, Die Quadratur des Kreises, S.4 [\EugenBeutelInt]}
