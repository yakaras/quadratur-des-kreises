\subsection{Bücher und ePrints}
[\EugenBeutelInt] Beutel, Eugen (1913): Die Quadratur des Kreises. \par Verlag B.G.Teubner, Leipzig und  Berlin \newline\newline[\BabineczInt] Babinecz, Wolfram: Rund um Kreis und Kugel. \par ePrint in cleverpedia.de, o.J. \par Unter: https://cleverpedia.de/wp-content/uploads/Rund-um-Kreis-und-Kugel.pdf \par Aufgerufen am: 28.10.2022 \newline\newline
[\FritschInt] Fritsch, Rudolf (1988): Transzendenz von e im Leistungskurs? \par ePrint in Uni Hamburg, Kiel \par Unter: \begin{simplechar}https://www.math.uni-hamburg.de/home/wockel/teaching/data\par 
 /AlgGeomStrSS2014/Transzendenz_e.pdf\end{simplechar} \par Aufgerufen am 31.10.2022 \newline\newline
[\ActaInt] Kochanski, Adam (1685): Adami Adamandi e Societ Jesu. \par Artikel in der Zeitschrift Acta Eruditorum\par Hrsg. J. Grossium et J.F. Gletitschium, Leipzig \par Unter: https://archive.org/details/s1id13206510/mode/2up \par Aufgerufen am 31.10.2022 \newline\newline
[\AlgebraInt] Hartlieb, Silke und Unger, Luise: Algebra und ihre Anwendungen. \par ePrint in Uni Hagen \par Unter: \begin{simplechar}
    https://www.fernuni-hagen.de/mi/studium/module/pdf/\par Leseprobe-komplett_01320.pdf
\end{simplechar}\par Aufgerufen am: 31.10.2022
\newline\newline
[\HobsonInt] Hobson, Ernest (1913): Squaring the circle. \par  Cambridge University Press, Cambridge \newline\newline
[\AnnalenInt] Klein Felix und Mayer Adolph (1882): Mathematische Annalen, Band. 20. \par Verlag B.G.Teubner, Leipzig\newline\newline[\MillaInt] Milla, Lorenz (2020$^3$): Die Transzendenz von \(\pi\) und die Quadratur des Kreises. \par
ePrint arXiv, DOI: 2003.14035\newline\newline
[\MohrInt] Mohr, Richard (2009): Die transzendente Zahl \(\pi\); normal? \par ePrint in Hochschule Essingen \par Unter:\begin{simplechar} https://www2.hs-esslingen.de/~mohr/praes/tot/pi/ziffern_pi.pdf \par Aufgerufen am: 31.10.2022\end{simplechar} \newline\newline
[\MoserInt] Moser, Lukas-Fabian (2013): Die Transzendenz der Zahlen $e$ und $\pi$ nach
Hilbert. \par\, ePrint in Uni München \par\, Unter: https://www.mathematik.uni-muenchen.de/~lfmoser/ss13/transzendenz.pdf \par\, Aufgerufen am: 15.10.2022\newline\newline
\newpage\noindent[\WieleitnerInt] Wieleitner, Heinrich (1927): Der Begriff der Zahl Band 2. \par\, Verlag B.G.Teubner, Leipzig und Berlin \newline\newline
[\PiUlm] Zoller, Janosch (2016): Die transzendente Zahl Pi. \par\, ePrint in Universität Ulm \par\, Unter: \begin{simplechar}
    https://dlc.campingrider.de/transz_pi_vor_mathematikgeschichte.pdf
\end{simplechar} \par\, Aufgerufen am: 31.10.2022\subsection{Internetquellen}
[\UniPresentationInt] Kohlhaase, Jan (2014): Vortrag: Die Quadratur des Kreises. \par\, Fakultät für Mathematik
Universität Duisburg-Essen \par\, \begin{simplechar}Unter: https://www.uni-due.de/mathematik/kohlhaase/lehre/tdot/Quadratur_des_\par\, Kreises.pdf \par\, Aufgerufen am: 15.10.2022 \end{simplechar}
\newline\newline
[\GoogleCloudInt]  Iwao, Emma (2022): A bigger piece of the pi: Finding the 100-trillionth digit. \par\, Google Cloud \par\, Unter: https://blog.google/products/google-cloud/new-digit-pi-2022/ \par\, Aufgerufen am 22.10.2022
\subsection{Bildquellen}
Abbildung 1,3,4, sowie Abbildungen aus \ref{kochanski_section}: Uhl, Johann, Geogebra \newline
sssAbbildung 2: Hobson, Ernest: Squaring the circle, S.15 [\HobsonInt]\newline
Abbildung \ref{qudr_circ}: Babinecz, Wolfram: Rund um Kreis und Kugel, S.12 [\BabineczInt]\newline
Abbildung 6: Hobson, Ernest: Squaring the circle, S.34 [\HobsonInt]
